  配置:
    name
         git config --global user.name "huanghai"
    email
         git config --global user.email "1991427903@qq.com"

使用 git:
    git status
        查看当前仓库的状态
    git init
        初始化仓库

    刚刚添加到项目中的文件处于未跟踪状态
        未跟踪 ---> 暂存
            git add <filename> 将文件切换到暂存的状态

            git add * 将所有已修改 (未跟踪) 的文件暂存

        暂存 ---> 未修改
            git commit -m "xxxx"  将暂存的文件存储到仓库中

            git commit -a -m "xxxxx"   提交所有已修改的文件     (未跟踪的文件不会提交)

        未修改 ---> 修改 
            修改代码后,文件会变为修改状态

        常见的命令:
        1.重置文件
            git restore <filename>   使状态变为之前的状态

            git restore --staged <filename>  取消暂存状态

        2.删除文件 

            git rm  <filename>  删除文件

            git rm  <filename> -f  强制删除文件

        3. 移动文件

            git mv  (to)  移动文件 重命名文件

        分支

        git 在储存文件时,每一次代码的提交都会创建一个与之对应的节点,
        git 就是通过一个一个节点来记录代码状态的。
        节点会构成一个树状结构,树状结构就意味着这个树会存在分支,默认情况下
        仓库只有一个分支,命名为master。在使用git 时,可以创建多个分支,分支与分支之间相互独立,
        在一个分支上修改代码不会影响其它分支。

        git branch 查看当前分支
        git branch name  创建新分支
        git branch -d name   删除分支
        git switch name  切换分支
        git switch -c name      创建并切换分支

        在开发中,都是在自己的分支上编写代码,代码编写完成后,再将自己的分支合并到主分支中

        git merge name   合并分支

        变基 (rebase)

        在开发中除了通过merge来合并分支外,还可以通过变基来完成分支的和并
        我们通过merge 合并分支时,在提交记录中会将所有分支创建和合并的过程全部显示出来,这样项目比较复杂,
        开发过程比较波折时,我们必须要反复创建、合并、删除分支。这样一来将会使得我们代码的提交记录变得极为混乱。

        git rebase name

        原理:
            1. 当我们发起变基时,git会首先找到两条分支最近的共同祖先
            2. 对比当前分支相对于祖先的历史提交,并且将他们提取出来储存到一个临时文件中
            3. 将当前主分支指向,指定的节点
            4. 根据要求保留节点
            5. 再执行一次快速合并

        远程仓库 (remode)

        远程git仓库和本地仓库本质没有什么区别不同点在于远程仓库可以被多人同时访问使用,方便协同开发。
        在实际工作中,git 的服务器通常由公司搭建内部使用或是购买一些公共的私有git服务器。
        我们学习阶段,直接使用一些开放的公共git仓库。目前我们常用的库有两个: GitHub 和 Gitee (码云)

        将本地库上传 git

        git remote add origin https://github.com/huanghai7137/gitdemo.git

        git branch -M main      修改分支的名字

        git push -u origin main

        将本地库上传 Gitee

        cd existing_git_repo

        git remote add origin https://gitee.com/taikule/git-demo.git
        
        git push -u origin "master"

        远程库的操作命令

        git remote      列出当前的关联远程库

        git remote add 命名远程库名 url   关联远程仓库

        git remote remove 远程库名      切断和远程库的联系

        git push 远程库名 本地分支:远程分支    把文件推送到指定远程分支上

        git clone url   从远程库下载代码

        git push    如果本地的版本低于远程库,push默认推不上去

        git fetch     要想推送成功,必须先确保本地库和远程库版本一致,fetch它会从远程仓库
                      下载所有代码,但是他不会将代码和当前分支自动合并

        git pull        从服务器上拉取代码并自动合并

        注意:  推送代码之前,一定要先从远程库中拉取最新代码

        克隆代码:
            git clone 下载链接

        tag  标签

        当头(HEAD)指针没有指向某个分支时,这种状态称为分离头指针(HEAD detached),分离头指针的状态下也可以操作代码,
        但是这些操作不会出现在任何分支上,所以不要在分离头指针的状态下来操作仓库。

        可以在创建新分支后再操作

        git switch -c 分支名  提交id   在指定节点创建分支

        git switch -c 分支名  节点标签  在指定节点创建分支

        可以为提交记录设置标签,设置标签以后,可以通过标签快速识别出不同的开发节点:

        git tag

        git tag 版本

        git tag 版本 提交id

        git push 远程库名 节点标签

        git push --tags     推送所有标签代表的节点

        git tag -d 标签名   删除标签

        git push 远程仓库 --delete 标签名   删除远程标签

        gitignore

        默认情况下,git会监视项目中所以有内容,但是有些文件我们没必要发到远程库中,可以在
        项目目录中创建一个  .gitignore 文件, 来设置需要忽略的文件。

        github 的静态页面

        在github 中,可以将自己的静态页面部署到github中,它会给我们提供一个地址使得我们的页面变成一个真正的网站,可以供用户访问。
        
        要求:
            静态页面的分支必须叫做: gh-pages
            如果希望页面可以通过xxx.github.io 访问,则需要将库的名字配置为  用户名.github.io

        
        docusaurus

        facebook推出的开源静态的内容管理系统,通过它可以快速的部署一个静态网站

        使用:
            网址:
                https://docusaurus.io/
            安装:
                npx create-docusaurus@latest my-website classic
            启动项目:
                npm start 
            构建项目:
                npm run build
            配置项目:
                docusaurus.config.js    项目的配置文件

            推送到github上:
                npm run deploy

            添加页面
                在docusaurus框架中, 页面分为三种: 1. page 2. blog   3.doc
            
        




            